\documentclass[11pt,a4paper,czech]{article}
\input{files/styles/header.tex}
\usepackage[margin=2.5cm,footskip=1cm,bottom=4cm,top=4cm]{geometry}

% nastavení fontů
\setmainfont[
    Path = files/fonts/StempelGaramond/,
    Ligatures=TeX,
    UprightFont = *-Roman.otf,
    BoldFont=*-Bold.otf,
    ItalicFont=*-Italic.otf,
    BoldItalicFont=*-BoldItalic.otf,
    BoldItalicFont=*-BoldItalic]{StempelGaramondCE}

% Pozadí stránky

% Číslování nadpisů

%%%%%%%%%%% DEFINOVÁNÍ STYLŮ %%%%%%%%%%%%%%%%%
\renewcommand{\labelitemi}{$-$}

\usepackage{babel}
\usepackage{array}
\usepackage{varioref}
\usepackage{booktabs}
\usepackage{url}
\usepackage{multirow}
\usepackage{setspace}


%%%%%%%%%%%%%%%%%%%%%%%%%%%%%% LyX specific LaTeX commands.
%% Because html converters don't know tabularnewline
\providecommand{\tabularnewline}{\\}

\usepackage{textcase}
\renewcommand{\textsc}[1]{\MakeTextUppercase{#1}} % příjmení autorů velkými písmeny

%% typografická zlepšení
\usepackage{csquotes}

%% floats get barriers
\usepackage[section]{placeins}

%% poznámky pod čarou
\usepackage{footnote} % možnost používat poznámky pod čarou v~tabulkách
\makesavenoteenv{tabular} % automaticky se budou ukládat poznámky pod čarou v~prostředí tabular
\makesavenoteenv{table} % automaticky se budou ukládat poznámky pod čarou v~prostředí table
% zůstává problém, že při kliknutí na odkaz na poznámku pod čarou vytvořený pomocí hyperref nefunguje

\usepackage{xunicode}

\input{../folder.tex}

\begin{document}

\from[žadatel]{\zadatel{full}}

\recipient[nadřízený správní orgán]{Ministr kultury \\ \nadrizenyspravniorgan{full}}

%\def \yoursign { }
%\def \oursign { }
\def \yourdate {17. 1. 2014}
\def \place {Praha}

\printheader

\subject{Žádost o uplatnění opatření proti nečinnosti Ministerstva kultury}

Vážený pane ministře,

podle § 80 odst. 3 zákona č. 500/2004 Sb., správního řádu, ve znění pozdějších předpisů (dále jen „SpŘ“) podávám {\bf žádost o uplatnění opatření proti nečinnosti} ministerstva. Ministerstvo bylo nečinné při vyřizování stížnosti na postup při vyřizování žádosti o informace. Má žádost se týkala {\emph \casename} 

Žádost o informace jsem podal povinnému subjektu \povinnysubjekt{short} dne 2. 1. 2014. Povinný subjekt poskytl informace jen částečně sdělením ze dne {\yourdate}, {\yoursign}. Stížnost na postup při vyřizování žádosti o informace jsem podal dne 22. 1. 2014. Přesto nadřízený správní orgán o stížnosti dodnes nijak nerozhodl. 

V souladu s § 80 odst. 4 písm. a) SpŘ navrhuji, aby nadřízený správní orgán {\bf přikázal} nečinnému správnímu orgánu, aby vydal rozhodnutí do 5 dnů od doručení příkazu. 

\signature{S úctou

\bigskip
\zadatel{short}}

\attachments

Žádost o informace \\
Stížnost na postup při vyřizování žádosti o informace

\end{document}
