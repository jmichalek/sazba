%% LyX 2.0.6 created this file.  For more info, see http://www.lyx.org/.
%% Do not edit unless you really know what you are doing.
\documentclass[11pt,a4paper,czech]{article}
\usepackage[T1]{fontenc}
\setcounter{tocdepth}{2}
\setlength{\parskip}{\medskipamount}
\setlength{\parindent}{0pt}
\usepackage{babel}
\usepackage{array}
\usepackage{calc}
\usepackage{url}
\usepackage[unicode=true,
 bookmarks=true,bookmarksnumbered=false,bookmarksopen=false,
 breaklinks=true,pdfborder={0 0 1},backref=false,colorlinks=false]
 {hyperref}
\hypersetup{pdftitle={Povinné zveřejňování informací (diplomová práce)},
 pdfauthor={Jakub Michálek},
 pdfsubject={zákon o svobodném přístupu k informacím},
 pdfkeywords={informace, publikační povinnost}}

\makeatletter

%%%%%%%%%%%%%%%%%%%%%%%%%%%%%% LyX specific LaTeX commands.
\pdfpageheight\paperheight
\pdfpagewidth\paperwidth

%% Because html converters don't know tabularnewline
\providecommand{\tabularnewline}{\\}

%%%%%%%%%%%%%%%%%%%%%%%%%%%%%% Textclass specific LaTeX commands.
\newenvironment{lyxlist}[1]
{\begin{list}{}
{\settowidth{\labelwidth}{#1}
 \setlength{\leftmargin}{\labelwidth}
 \addtolength{\leftmargin}{\labelsep}
 \renewcommand{\makelabel}[1]{##1\hfil}}}
{\end{list}}

%%%%%%%%%%%%%%%%%%%%%%%%%%%%%% User specified LaTeX commands.
%% fonty
\usepackage{fontspec}
\setmainfont[Ligatures=TeX]{Stempel Garamond}
\defaultfontfeatures{Mapping=tex-text}
\usepackage{textcase}
\renewcommand{\textsc}[1]{\MakeTextUppercase{#1}} % příjmení autorů velkými písmeny

%% typografická zlepšení
%\usepackage{csquotes}

\usepackage{titlesec}

\titleformat{\paragraph}[display]
{\normalfont\bfseries}{Čl.~\arabic{paragraph}}{1em}{}

%% floats get barriers
\usepackage[section]{placeins}

%% poznámky pod čarou
\usepackage{footnote} % možnost používat poznámky pod čarou v tabulkách
\makesavenoteenv{tabular} % automaticky se budou ukládat poznámky pod čarou v prostředí tabular
\makesavenoteenv{table} % automaticky se budou ukládat poznámky pod čarou v prostředí table
% zůstává problém, že při kliknutí na odkaz na poznámku pod čarou vytvořený pomocí hyperref nefunguje

%% citování
\usepackage[backend=biber,
style=footnote-dw,%authortitle-dw
namefont=smallcaps,
isbn=true,
language=english,
autocite=footnote,
backref=true,
hyperref,
nopublisher=false]{biblatex} 
\urlstyle{rm}

\DeclareFieldFormat{title}{\mkbibemph{#1}} % titul kurzívou
\let\cite\autocite

\defbibheading{bibliography}{\section*{Seznam použité literatury}} % přejmenování sekce

%% závěrečné úkony
\AtEndDocument{
\addcontentsline{toc}{section}{Seznam použité literatury}
\printbibliography} % tisk bibliografie na konci souboru

\let\finalandcomma=\!
\renewcommand*{\multinamedelim}{\addcomma\space}
\renewcommand*{\finalnamedelim}{%
\ifnum\value{liststop}>2 \finalandcomma\fi%
\addsemicolon\space}
\renewcommand*{\bibmultinamedelim}{\addcomma\space}
\renewcommand*{\bibfinalnamedelim}{%
\ifnum\value{liststop}>2 \finalandcomma\fi%
\addsemicolon\space}%
\renewcommand*{\citemultinamedelim}{\addsemicolon\space}
\renewcommand*{\citefinalnamedelim}{\addsemicolon\space}
\renewcommand*{\labelnamepunct}{\addperiod\space}
\renewcommand*{\nametitledelim}{\addperiod\space}
\renewcommand*{\newunitpunct}{\addperiod\space}

\DefineBibliographyStrings{english}{%
seenote          = {viz pozn\adddot},
page          = {s\adddot},
pages          = {s\adddot},
backrefpage = {citováno na s\adddot},
backrefpages= {citováno na s\adddot},
url = {Dostupný z WWW\addcolon},
byeditor         = {ed\adddotspace},
urlseen         = {cit\adddot},
}

% popiseky tabulek
\usepackage[hang,bf,small]{caption} % úprava popisku tabulky
\setlength{\captionmargin}{20pt}

% tisk částí
\let\stdpart\part
\renewcommand*{\part}{\clearpage\stdpart}

%% nechceme zobrazovat url v dokumentu
%\AtEveryCite{%
%\clearfield{url}%
%\clearfield{urldate}%
 %\DeclareFieldFormat{url}{}%
 %\DeclareFieldFormat{urldate}{}%
%}

% Formát data v url
\DeclareFieldFormat{urldate}{%
  \iffieldundef{urlday}
    {}
    {\stripzeros{\thefield{urlday}}\adddot%
\stripzeros{\thefield{urlmonth}}\adddot%
\printfield{urlyear}}%
}

\renewbibmacro{url+urldate}{%
\iffieldundef{urlday}
    {}
  {\space\printtext{[}\bibstring{urlseen} \printurldate\printtext{]}} 
\iffieldundef{url}
{}
{\printfield{url}\adddotspace}}

\DeclareFieldFormat{url}{\bibstring{url} <\url{#1}>}

% Správné zpětné citace
\renewcommand*{\finentrypunct}{}
\renewbibmacro*{pageref}{%
  \addperiod% NEW
  \iflistundef{pageref}
    {}
    {\printtext[parens]{% NEW
       \ifnumgreater{\value{pageref}}{1}
         {\bibstring{backrefpages}\ppspace}
     {\bibstring{backrefpage}\ppspace}%
       \printlist[pageref][-\value{listtotal}]{pageref}\addperiod}}}% NEW

\bibliography{literature/bibliography}

\AtBeginDocument{
  \def\labelitemi{\normalfont\bfseries{--}}
}

\makeatother

\usepackage{xunicode}
\begin{document}

\title{Zaměstnanecká díla na školách}


\author{Mgr. Bc. Jakub Michálek%
\thanks{Absolvent Právnické fakulty UK a Matematicko-fyzikální fakulty UK,
člen Ediční komise AS UK. Pracuje jako advokátní koncipient ve společnosti
Řehák \& Partneři, k.\,s., Praha.%
}}
\maketitle
\begin{abstract}
Podnikatelé využívají svých oprávnění podle autorského zákona za účelem
dosažení zisku. Naproti tomu školy a jiné veřejné instituce svá práva
podle autorského zákona často zanedbávají, což se může nepříznivě
projevit na jejich účelu -- šíření vzdělanosti. Na školách se vytváří
zejména díla školní (studenti) a zaměstnanecká (učitelé). Povědomí
o zaměstnaneckých dílech není v České republice příliš rozšířené.
Ke zlepšení této situace jsem vytvořil stručného průvodce, který by
měl představitelům vysokých, středních a základních škol pomoci při
pochopení právní úpravy zaměstnaneckých děl a při jejím efektivním
využívání ku prospěchu vzdělanosti. 
\end{abstract}
\newpage{}\tableofcontents{}\newpage{}


\section*{Právní stav zpracování publikace}

Text je zpracován k právnímu stavu ke dni \today{}. Současná úprava
zaměstnaneckých děl účinkuje od 1. 12. 2000. Před tímto datem byly
poměry při vzniku zaměstnaneckého díla podstatně odlišné, a tedy informace
uvedené v tomto textu se na dříve vytvořená díla nevztahují.


\subsection*{Seznam zkratek použitých předpisů}
\begin{lyxlist}{00.00.0000}
\item [{\textbf{AZ}}] zákon č. 121/2000 Sb., autorský zákon
\item [{\textbf{NOZ}}] zákon č. 89/2012 Sb., občanský zákoník
\item [{\textbf{ObčZ}}] zákon č. 40/1964 Sb., občanský zákoník
\item [{\textbf{ObchZ}}] zákon č. 513/1991 Sb., obchodní zákoník (zrušen)
\item [{\textbf{ZPr}}] zákon č. 262/2006 Sb., zákoník práce
\item [{\textbf{ZVŠ}}] zákon č. 111/1998 Sb., o vysokých školách
\item [{\textbf{ZVV}}] zákon č. 130/2002 Sb., o podpoře výzkumu a vývoje
z veřejných prostředků
\end{lyxlist}
Odkazu na již zrušený (přesto někdy použitelný) právní předpis uvádím
v hranatých závorkách za odkazem na platnou právní úpravu.

\newpage{}


\section{Zákonná úprava zaměstnaneckých děl}

Základní úpravu zaměstnaneckého díla a souvisejících otázek nalezneme
v autorském zákoně vydaném pod č.~121/2000 Sb., ve znění pozdějších
předpisů (dále též zkráceně „AZ“).


\subsection{Co je to zaměstnanecké dílo?}

\marginpar{\textbf{zaměstnanecké dílo}

§~58 odst. 1 AZ}%
\framebox{\begin{minipage}[t]{1\columnwidth}%
\textbf{Zaměstnanecké dílo} je autorské dílo, které autor vytvořil
ke splnění svých povinností vyplývajících z pracovněprávního vztahu
ke svému zaměstnavateli. %
\end{minipage}}

Za těchto předpokladů budou zaměstnaneckým dílem zpravidla
\begin{itemize}
\item pracovní list, který učitel napsal na počítači při přípravě na hodinu,
\item skripta napsaná vysokoškolským učitelem pro studenty, pokud je učitel
v mzdové třídě zahrnující publikační činnost vzdělávacích materiálů,
\item vědecký článek nebo jiný výsledek tvůrčí vědecké práce, 
\item habilitační práce akademické akademického pracovní pracovníka ke splnění
požadavku zaměstnavatele na řádný výkon práce,\cite{Telec2008}
\item fotografická dokumentace školních akcí provedená učitelem,
\item přednáška připravovaná učitelem a prezentace pro ni vytvořená, 
\item posudek dizertační nebo jiné kvalifikační práce,\cite{Telec2008}
\item příspěvek učitele vkládaný při jeho práci do systému jako jsou Wikiskipta
apod.
\end{itemize}
Poznámky:
\begin{itemize}
\item Autorské dílo\marginpar{§~2 odst. 1 AZ} je chráněno \textbf{bez
ohledu na rozsah nebo význam}. Může jít jen o několik slov (např.
editace encyklopedie Wikipedia je zpravidla také autorským dílem,
pokud vyplývá z povinnosti vyučujícího, pak i dílem zaměstnaneckým).
\item Není podstatné, zda je práce vykonávána v pracovní době nebo prostředky
zaměstnavatele, nýbrž pouze to, zda ji zaměstnanec vytvořil ke splnění
svých povinností vyplývajících z pracovněprávního vztahu. Pracovní
pokyny lze dohodnout ve smlouvě nebo je může zaměstnavatel určit v
rámci sjednaného druhu práce jednostranně buď konkrétním úkolem nebo
obecně určením pracovní náplně (např. ve vnitřním mzdovém předpisu\cite[s. 64]{Telec2008}). 
\item Není významné, zda druh práce (funkce zaměstnance), sjednaný v pracovní
smlouvě, zní výslovně na tvůrčí práci literární, uměleckou nebo vědeckou.
Podstatné je, že z ní alespoň potenciálně vyplývá, že jejím výsledkem
může být vznik díla literárního, uměleckého nebo vědeckého.%
\footnote{Rozhodnutí Obvodního soudu pro Prahu 1, sp. zn. 14 C 4/1982, 14 C
5/1982 a 14 C 6/1982.%
}
\end{itemize}
Za zaměstnanecká díla se se všemi důsledky s tím spojenými považují
také:
\begin{itemize}
\item \marginpar{§~58 odst.~1 AZ}díla vytvořená ke splnění povinností
vyplývajících z pracovního vztahu autora k družstvu či ze služebního
vztahu,
\item \marginpar{§~58 odst.~10 AZ}díla vytvořená ke splnění povinnosti
vyplývajících ze vztahu mezi právnickou osobou a autorem, který je
členem jejího orgánu,
\item \marginpar{§~58 odst.~7 AZ}počítačové programy, databáze a kartografická
díla vytvořené autorem na objednávku (§~58 odst.~7 AZ),
\item \marginpar{§~59 odst.~2 AZ}kolektivní díla%
\footnote{Kolektivním dílem je dílo, na jehož tvorbě se podílí více autorů,
které je vytvářeno z podnětu a pod vedením fyzické nebo právnické
osoby a uváděno na veřejnost pod jejím jménem, přičemž příspěvky zahrnuté
do takového díla nejsou schopny samostatného užití. Kolektivním dílem
však nejsou díla audiovizuální a díla audiovizuálně užitá. Za podmínek
věty první budou kolektivním dílem např. mapy nebo počítačové programy.%
} vytvořená spoluautory na objednávku.
\end{itemize}
Autorem ve smyslu autorského zákona je vždy člověk -- nemůže jím být
obchodní společnost nebo jiná právnická osoba. Ta může být pouze nositelem
práv, o čemž bude řeč dále.


\subsection{Výkon majetkových práv k zaměstnaneckému dílu}

\marginpar{§~58 AZ}Základní zásadu výkonu autorských práv lze shrnout
takto:

\framebox{\begin{minipage}[t]{1\columnwidth}%
Majetková práva k zaměstnaneckému dílu vykonává zaměstnavatel. %
\end{minipage}}

To znamená, že ve věci majetkových autorských práv k zaměstnaneckému
dílu jedná zaměstnavatel svým jménem a na svůj účet. Autor může dílo
užít pouze se souhlasem zaměstnavatele. Zaměstnavatel získává oprávnění
vykonávat majetková práva k zaměstnaneckému dílu, a to i vůči autorovi\marginpar{§~12 AZ}.
Zaměstnavatel má právo
\begin{itemize}
\item \marginpar{§~12 odst. 1 AZ}zaměstnanecké dílo užít%
\footnote{Obsah práva dílo užít je vymezen příklady v §~13 až §~23 AZ.%
} v původní nebo jiným zpracované či jinak změněné podobě,
\item \marginpar{§~12 odst. 1 AZ}udělit jinému oprávnění licenční smlouvou;
bez tohoto oprávnění může autor a jiná osoba dílo užít pouze v případech,
které stanoví zákon,
\item \marginpar{§~12 odst. 3 AZ}žádat po autorovi nebo jiné osobě, která
má originál díla v držení, aby mu ho zpřístupnil nebo na náklady zaměstnavatele
zhotovil kopii a poskytl ji zaměstnavateli,
\item \marginpar{§~30 a 30a AZ}na odměnu v souvislosti s rozmnožováním
zaměstnaneckých děl pro osobní potřebu; zaměstnavatel by se za účelem
tohoto práva měl přihlásit u příslušného kolektivního správce, který
vybírá poplatky od uživatelů. 
\end{itemize}
Osobnostní práva autorská uvedená v §~11 AZ zůstávají autorovi zachována.
Zaměstnavatel však smí, pokud se se zaměstnancem nedohodl jinak, dílo
zveřejnit, upravit, zpracovat, spojit s jiným dílem, zařadit do souboru,
uvádět dílo pod svým jménem nebo dílo dokončit, pokud ho zaměstnanec
nedokončil.

Výjimky ze zásady, že majetková práva k zaměstnaneckému dílu vykonává
zaměstnavatel:
\begin{itemize}
\item Zaměstnanec\marginpar{§~58 odst. 1 AZ} a zaměstnavatel se mohou
dohodnout, že autorská práva majetková nebude vykonávat zaměstnavatel,
ale zaměstnanec svým jménem a na svůj účet (souhlasit tedy musí jak
zaměstnanec, tak zaměstnavatel).
\item Zaměstnavatel může právo výkonu majetkových práv k zaměstnaneckému
dílu postoupit třetí osobě, ale pouze se svolením autora. Pokud se
však prodává obchodní závod (dříve podnik), \marginpar{§ 58 odst. 1 AZ\protect \\
§ 2178 NOZ\protect \\
dříve §~479 ObchZ}právo k výkonu majetkových práv k zaměstnaneckému dílu přechází společně
s obchodním závodem, ledaže původní pracovní smlouva nebo kupní smlouva
na obchodní závod určí jinak.
\item Na úřední díla a tradiční výtvorů\marginpar{§~3 AZ} lidové kultury
se ustanovení zákona o majetkových právech autorských vůbec nepoužijí.
Šíření takových děl autorský zákon neomezuje; je však třeba přihlédnout
k jiným zákonům.
\end{itemize}
Poznámky:
\begin{itemize}
\item K užití zaměstnaneckého díla zaměstnavatelem není potřeba uzavírat
samostatnou smlouvu, neboť zaměstnavatel vykonává majetková autorská
práva ze zákona.
\item Ve všech ostatních případech je k užití zaměstnaneckého díla třetí
osobou potřeba uzavřít licenční smlouvu se zaměstnavatelem, ledaže
zákon stanoví pro takové užití výjimku.
\item Tvůrčím\textbf{ zpracováním} \marginpar{\textbf{zpracování}\protect \\
§~2 odst. 4 AZ}autorského díla do vlastního díla není dotčeno právo původního nositele
práv. Pokud tedy zpracujeme zaměstnanecké dílo a toto zpracování chceme
dále šířit, musíme mít souhlas zaměstnavatele.
\end{itemize}
Nedohodnou-li se zaměstnavatel a zaměstnanec jinak, zaměstnanci zbývá
k zaměstnaneckému dílu jen ,,holé autorství``,\cite{komentar}\marginpar{\textbf{holé autorství}

§~58 odst. 2, 3, 6 AZ} které spočívá zejména v tom, že 
\begin{itemize}
\item \marginpar{§~58 odst. 2 AZ}zaměstnanec v případě zániku zaměstnavatele
bez právního nástupce získává výkon majetkových práv autor zpět,%
\footnote{Jde o tzv. konsolidační zásadu,\cite{komentar} která je projevem
tzv. elasticity vlastnictví. %
}
\item \marginpar{§~58 odst. 3 AZ}zaměstnanec má právo požadovat od zaměstnavatele
udělení licence za obvyklých podmínek, pokud zaměstnavatel nevykonává
majetková práva k zaměstnaneckému dílu vůbec nebo jen nedostatečně,
ledaže má zaměstnavatel závažný důvod k jejímu odmítnutí,
\item \marginpar{§~58 odst. 6 AZ}zaměstnanec má právo na přiměřenou dodatečnou
odměnu, pokud se jeho mzda nebo odměna dostane do zjevného nepoměru
k tomu, nakolik zaměstnavatel dílo zpeněžil; to však neplatí, pokud
se dohodli ve smlouvě jinak.
\end{itemize}
Právní úpravu zaměstnaneckých děl, u kterých se zaměstnavatel a zaměstnanec
nedohodli jinak, lze shrnout do tab.~\ref{tab:Uprava-zam-del}.

\begin{table}
\begin{tabular*}{1\textwidth}{@{\extracolsep{\fill}}|>{\raggedright}p{0.25\textwidth}|>{\raggedright}p{0.24\textwidth}|>{\raggedright}p{0.18\textwidth}|>{\raggedright}p{0.2\textwidth}|}
\hline 
Okruh práv k~zaměstnaneckému dílu & Obsah práva & Nositel práva & Převoditelnost\tabularnewline
\hline 
\hline 
\textbf{Majetková práva autorská} \\
(§ 12 a násl. AZ) & holé autorství \\
(viz výše) & zaměstnanec \\
(autor díla) & jen dědění\\
§~26 AZ\tabularnewline
\hline 
\textbf{Právo na výkon majetkových práv autorských}\\
(§ 58 AZ) & užití díla všemi způsoby

udělení licence jiné osobě (§~46 AZ) & zaměstnavatel & za podmínek\\
§~58 odst. 1 AZ\tabularnewline
\hline 
\textbf{Právo dílo užít na základě licence}\\
(§~12 odst. 1 AZ) & užití díla způsobem uvedeným v~licenci & zaměstnavatel\\
oprávněný uživatel & za podmínek \\
§~48 AZ\tabularnewline
\hline 
\end{tabular*}

\caption{\label{tab:Uprava-zam-del}Právní úprava zaměstnaneckých děl, pokud
se zaměstnavatel a zaměstnanec nedohodnou jinak.}
\end{table}



\subsection{Další případy omezení výkonu majetkových práv autorem}

Zaměstnanecká díla jsou pouze jedním z případů, kdy je právo autora
užít dílo a udělit licenční smlouvou jinému zákonem omezeno. Dalším
takovým příkladem jsou výhradní licence,\marginpar{\textbf{výhradní licence}

§~47 odst.~2 AZ} po jejichž udělení se autor musí zdržet užití díla, nedohodne-li
se jinak. 

\textbf{Nositelem práv} \marginpar{\textbf{nositel práv}\protect \\
§~95 odst.~2 AZ}nazýváme fyzickou nebo právnickou osobu, 
\begin{itemize}
\item které přísluší majetkové právo autorské, 
\item která je ze zákona oprávněna vykonávat právo dílo užít (např. výše
uvedený zaměstnavatel) nebo 
\item která má výhradní licenci a může udělit jiné osobě oprávnění k užití
díla.
\end{itemize}
Zvláštním případem, kdy autor nebo jiný nositel práv nemůže své právo
vykonávat sám, je zákonné zastoupení autora nebo nositele práv v případě
\marginpar{§~96 AZ}povinně kolektivně spravovaných práv. 

V případě zaměstnaneckých děl autor dostává odměnu v podobě mzdy nebo
jiné odměny, v případě výhradní licence je odměna stanovena v licenční
smlouvě, v~případě povinné kolektivní správy práv dostává autor odměnu
od kolektivního správce (např. z poplatku, který kolektivní správce
vybírá od výrobců nosičů nebo tiskáren).

Pokud vykonává autorská práva k zaměstnaneckému dílu zaměstnavatel,
je také oprávněn na svůj účet vykonávat majetkové právo autora na
náhradní odměnu\marginpar{\textbf{náhradní odměny}

§~25 AZ} za užití. Univerzitě, jiné veřejné vysoké škole či jiné veřejné instituci,
která vykonává majetková práva k velkému souboru zaměstnaneckých děl,
lze doporučit, aby se za tímto účelem přihlásila u příslušného kolektivního
správce (u literárních děl jde o občanské sdružení Dilia). 


\subsection{Práva k výsledkům vytvořeným za veřejné podpory ve výzkumu, vývoji
a inovacích}

Práva k autorským dílům, která vzniknou při výzkumu a vývoji podporovaném
z veřejných prostředků upravuje zákon č. 130/2002 Sb., o podpoře výzkumu
a vývoje z veřejných prostředků, ve znění pozdějších předpisů (dále
také zkráceně „ZVV“).

Práva k výsledkům výzkumu a vývoje patří buď příjemci veřejné podpory
nebo poskytovateli, přičemž rozhodující je, zda jde o \textbf{výsledek
veřejné zakázky}. 
\begin{itemize}
\item \marginpar{§ 16 odst. 2 ZVV}Jde-li o výsledek veřejné zakázky, je
příjemce, pokud poskytovatel nestanoví jinak, povinen udělené vlastnické
právo převést na poskytovatele (zadavatele). V případě zaměstnaneckých
děl jde o vlastnické právo k věci,%
\footnote{Podle § 489 NOZ je věcí v právním smyslu vše, co je rozdílné od osoby
a slouží potřebě lidí, tedy i právo, jehož povaha to připouští. %
} kterou je právo na výkon autorských práv. Tuto věc lze převést jen
se souhlasem autora, nicméně zaměstnavatel může udělit poskytovateli
výhradní licenci s právem udílet podlicence, což má z majetkového
hlediska obdobný účinek jako převod věci.
\item \marginpar{§ 16 odst. 3 ZVV}Nejde-li o výsledek veřejné zakázky,
patří práva příjemci. Příjemce, který není fyzickou osobou, upraví
způsob nakládání s výsledky svým vnitřním předpisem. To se bude typicky
týkat veřejných vysokých škol a právnických osob ve výzkumu a vzdělávání.
Zákon následně stanoví,\marginpar{§ 16 odst. 4 ZVV} že výsledek
plně financovaný z veřejných prostředků musí být přístupný za rovných
podmínek všem zájemcům o využití; další pravidla pro menší veřejnou
podporu stanoví odstupňovaně.
\end{itemize}
V případě zaměstnaneckého díla ve výzkumu platí to, co bylo dříve
uvedeno, tedy majetková práva autora vykonává ze zákona zaměstnavatel. 


\section{Úprava zaměstnaneckých děl zaměstnavatelem}

Tato část se věnuje následujícím otázkám: 
\begin{itemize}
\item Jaké problémy vznikají školám v důsledku chybného nebo nedostatečného
použití zákonné úpravy zaměstnaneckých děl?
\item Jakým způsobem školy v současné době přistupují k úpravě šíření zaměstnaneckých
děl?
\item Jaké je poslání škol a z jakých principů by vnitřní úprava zaměstnaneckých
děl měla vycházet?
\item Jakým způsobem v konkrétním případě upravit zaměstnanecká díla ve
školních poměrech?
\end{itemize}

\subsection{Praktické problémy}

Praktické problémy týkající se zaměstnaneckých děl lze shrnout do
následujících okruhů:
\begin{itemize}
\item \textbf{Neoprávněná užití}\\
Většina uživatelů autorských děl v populaci není dobře informována
o autorském právu, natož pak o ustanoveních o zaměstnaneckých dílech.
Tato nevědomost však nemění nic na tom, že pokud užijí autorské dílo
bez licenční smlouvy se zaměstnavatelem, může se zaměstnavatel domáhat
ochrany svého práva v souladu s §~41 autorského zákona u soudu, zejména
může žádat \marginpar{\textbf{bezdůvodné obohacení}

§~40 odst. 4 AZ} vydání bezdůvodného obohacení ve výši dvojnásobku obvyklé licenční
odměny. Porušování autorského práva je také \marginpar{§~105a odst. 1 písm. a) AZ}přestupkem.
Pokud by šlo o společensky škodlivý případ (například zaměstnanec
vydává zaměstnanecké dílo u komerční nakladatelství), může jít i o
\marginpar{§~270 TrZ}trestný čin porušování autorského práva. Tyto
delikty jsou příslušné orgány povinny stíhat, kdykoliv se o nich dozví. 
\item \textbf{Právní nejistota}\\
Pokud nejsou práva a povinnosti při užití autorských děl upravena
v souladu se zákonem, ocitají se jejich uživatelé v právní nejistotě.
Z důvodu právní nejistoty pak mohou odmítnout poskytnutí nebo užití
díla, které by bylo v souladu se zákonem (například poskytnutí jiné
škole k využití nad rámec výukové licence podle §~31 odst.~1 písm.~c)
AZ, umístění díla na Internet samostatně nebo v rámci souboru jako
je encyklopedie Wikipedie), nebo naopak uzavírají smlouvy, které jsou
ze zákona neplatné (například autor zaměstnaneckého díla neplatně
uděluje licenci komerčnímu nakladatelství, pokud nemá licenční smlouvu
se zaměstnavatelem).
\item \textbf{Nepřístupnost pro uživatele}\\
Pokud mají být zaměstnanecká díla legálně zpřístupněna žákům, studentům,
absolventům a veřejnosti, musí zaměstnavatel udělit licenci k jejich
užití. Pokud zaměstnavatel svá práva nevykonává, může se jejich svévolného
výkonu chopit zaměstnanec, například vydá knihu u externího nakladatelství
na základě smlouvy, která je od počátku neplatná. Jednání takového
zaměstnance je protiprávní, neboť zaměstnanec měl zaměstnavatele správně
požádat o udělení licence.\marginpar{§~58 odst.~3 AZ} K takto vydané
knize si zpravidla osobuje výhradní práva nakladatelství, které brání
zpřístupnění mimo papírovou knižní produkci, neboť právě z prodeje
výtisků má většinou největší zisk. Místo poslání školy tak majetková
práva autorská slouží komerčnímu zájmu, který se nesnaží o nejvyšší
dostupnost, ale o nejvyšší zisk, což je v rozporu se zájmem společnosti,
která tvorbu zaměstnaneckých děl na školách financuje.
\item \textbf{Nemožnost regulace}\\
Pokud škola nemá jasnou vnitřní úpravu zaměstnaneckých děl, nemůže
ani jejich tvorbu ovlivňovat, oceňovat učitele vytvářející nejlepší
studijní materiály, zpřístupňovat studijní materiály svým studentům
a veřejnosti, vydávat je vlastním jménem u nakladatelství, zvyšovat
svou prestiž ediční politikou a často ani nemůže efektivně sdílet
materiály mezi jednotlivými vyučujícími. U velkých univerzit je až
80~\% knih, z nichž velká většina jsou z převážné části zaměstnanecká
díla, vydáváno u externích nakladatelství.\cite[s. 4]{Stech2013}
Škola pak nemůže ovlivnit kvalitu vydávaných knih, která jsou spojována
s jejím jménem, ani jejich dostupnost pro studenty. Často pak na vysokých
školách dochází k situaci, kdy učitel požaduje po studentech, aby
si zakoupili ke zkoušce jeho knihu napsanou v pracovní době a vykázanou
v rámci publikační činnosti. Knihu přitom draze prodává soukromé komerční
nakladatelství a učiteli z jejího prodeje vyplácí podíl.
\end{itemize}

\subsection{Základní varianty právní úpravy}

V současné situaci lze rozeznávat následující možné přístupy škol
k zaměstnaneckým dílům:
\begin{itemize}
\item \textbf{Přehlížení zaměstnaneckých děl }\\
Škola neupravuje vytváření, evidenci či zpřístupnění zaměstnaneckých
děl. Míra šíření materiálů v takovém případě závisí na zaměstnancích,
kteří jsou jejich autory. Zpravidla jde o nepříliš výnosné publikace
malého rozsahu, u kterých se nevyplatí žádný jiný přístup. Anarchie
v oblasti výukových materiálů, které jsou zaměstnaneckými díly, je
typická pro základní a střední školy.
\item \textbf{Výkon práv k zaměstnaneckým dílům}\\
Škola upravuje nakládání se zaměstnaneckými díly ve svém vnitřním
předpise tak, že potvrzuje výkon majetkových práv k zaměstnaneckým
dílům uvedený v zákoně a zakládá pravidla pro evidenci těchto děl.
Podle povahy díla a dalších okolností je statutární orgán školy, případně
představitel její součásti tato práva oprávněn vykonávat. Tuto možnost
využívá většina univerzit.%
\footnote{Bod 3.4 opatření rektora UK č. 40/2009, o nakládání s výsledky výzkumu,
vývoje a inovací na Univerzitě Karlově v Praze. Dostupná z webu \url{http://www.cuni.cz/UK-3713.html}

Bod 4 směrnici rektora VŠE s názvem Ochrana a realizace práv duševního
vlastnictví jako výsledku výzkumu, vývoje a inovací nebo vytvořeného
jinak na VŠE (SR 01/2011). Dostupná z webu \url{http://www.vse.cz/predpisy/271} %
} Je však třeba dodat, že vysoké školy dodržování takového předpisu
často dostatečně neevidují, nekontrolují a nesankcionují. To způsobuje,
že někteří autoři svá zaměstnanecká díla neoprávněně zpeněžují u soukromých
nakladatelství, nicméně takové smlouvy jsou ze zákona neplatné.\marginpar{§ 580 odst. 1 NOZ\protect \\
{[}§~39 ObčZ{]}} \\
V rámci výkonu práv k zaměstnaneckým dílům lze rozlišovat dva základní
přístupy, které se však nemusí vyskytovat izolovaně, ale jsou na většině
škol kombinovány:

\begin{itemize}
\item \textbf{Zpeněžení} \textbf{zaměstnaneckých děl}\\
Škola zřizuje střediska pro zpeněžení autorských práv, patentových
práv a jiných majetkových práv. V případě patentových práv tato střediska
mohou vydělávat na faktickém postoupení těchto práv podnikatelům (například
průmyslovým závodům). V případě autorských práv škola vydává díla
vlastním nákladem nebo opravňuje za úplatu k vydání zaměstnaneckého
díla soukromé nakladatelství výhradní licencí. V případě školních
nakladatelství (např. Nakladatelství Karolinum v rámci UK) se však
často jedná jen o zpeněžení zdánlivé, neboť nadpoloviční příjem těchto
nakladatelství často tvoří dotace. Ve světě však existují u velkých
škol s mezinárodním portfóliem i zisková nakladatelství (např. univerzita
v Cambridge, UK%
\footnote{Podle výroční zprávy za rok 2013 mělo nakladatelství univerzity v
Cambridge celkový obrat 261,7 mil. liber, přičemž zisk před zdaněním
činil 8,2 mil. liber. Zdroj: \url{http://www.cambridge.org/about-us/who-we-are/annual-report1/chief-executives-overview/}%
}). 
\item \textbf{Otevřený přístup k zaměstnaneckým dílům}\\
Stejně tak může škola využít svých oprávnění, aby zaměstnanecké dílo
zpřístupnila veřejnosti, přičemž z takového zpřístupnění nemá škola
žádný majetkový či hospodářský prospěch. Naplňuje tím však lépe své
poslání šířit vzdělanost ve společnosti. K politice otevřeného přístupu
se přidaly v oblasti výzkumu přední světové univerzity.%
\footnote{Modelová politika Open Access. Dostupný z webové stránky \url{http://osc.hul.harvard.edu/modelpolicy}%
} V České republice se k otevřenému principu částečně i ve vztahu k
vzdělávacím materiálům přihlásila Univerzita Karlova v Praze,\cite[s. 8]{Stech2013}
na řadě vzdělávacích pracovišť je však dostupnost studijních materiálů
v elektronické podobě prakticky samozřejmostí. Některé univerzity
se přidaly k otevřenému přístupu také v oblasti vzdělávání a učebních
textů, přičemž z otevřené přístupu existují výjimky, o které může
tvůrce v konkrétním případě požádat.%
\footnote{Velice široký otevřený přístup má např. Polytechnická škola v Otago
na Novém Zélandě. Dostupný na webové stránce \url{http://wikieducator.org/Otago_Polytechnic/Intellectual_property}%
}
\end{itemize}
\item \textbf{Přenechání výkonu práv zaměstnanci}\\
Škola se fakticky vzdává svého výsadního postavení zaměstnavatele
ve vztahu k nakládání se zaměstnaneckými díly. Na některých předních
světových univerzitách je taková situace považována za tradiční právo
zaměstnanců a je jim přiznána vnitřním předpisem.%
\footnote{Porovnání předních světových univerzit. Dostupné na webu \url{http://web.mit.edu/committees/ip/policies.html}%
} K tomu je jen potřeba poznamenat, že tyto zahraniční školy jsou soukromé,
a tedy nelze se u nich odvolávat na právo občana na přístup k materiálům,
jejichž vznik z daní financuje. Podobný postup používá v České republice
Univerzita Palackého v Olomouci, která výkon práv přenechává autorovi.%
\footnote{Směrnice rektora UP č. B3-09/2-SR, o ediční činnosti Univerzity Paleckého
v Olomouci. Dostupný na webu \url{http://www.upol.cz/fileadmin/user_upload/dokumenty/SRB3-09-2_opravena_31032010.pdf}
Byť se tak neděje výslovnou dohodou, mám za to, že autor projevuje
svůj souhlas s tímto návrhem tím, že se podle něho zachová.%
} Přenechání výkonu práv zaměstnanci je však také kritizováno, pokud
se škola dobrovolně vzdává majetku, který byl vytvořen za peníze z
veřejných rozpočtů (narozdíl od zaměstnaneckých děl na předních univerzitách,
které jsou soukromé).
\end{itemize}
Příjemcem zisku z veřejného kapitálu by zásadně nemělo být soukromé
nakladatelství. Takové nakládání s majetkem může být v rozporu se
zásadou hospodárnosti \marginpar{§~20 odst. 1 ZVŠ} a využití majetku
k poslání školy. Pochybnosti vyvolává také fakt, že učitel rozhoduje
o povinné literatuře ke zkoušce, a ovlivňuje tak prodejnost vlastních
knih, za což pak platí studenti. 

Posláním škol je šíření vzdělanosti. Z výše uvedených variant mám
proto za to, že nejvhodnějším postupem je otevřený přístup k zaměstnaneckým
dílům, který bude v určitých zvláštních případech doplněn jejich zpeněžením.
Tímto přístupem se budu dále zabývat.


\subsection{Právní otázky vnitřní úpravy zaměstnavatelem}

Školy pouze zřídka využívají všech možností, které poskytuje zákonná
úprava zaměstnaneckých děl. V této části rozeberu klíčové otázky při
prosazování otevřeného přístupu k zaměstnaneckým dílům ve škole.


\subsubsection{Pracovněprávní vztahy}

Pracovněprávní vztahy upravuje zákon č. 262/2006 Sb., zákoník práce,
ve znění pozdějších předpisů (dále též zkráceně ,,ZPr``). Zaměstnanec
je v pracovněprávním vztahu povinen\marginpar{§ 2 odst. 1 ZPr} plnit
pokyny zaměstnance. V pracovní smlouvě je typicky vymezen pouze druh
práce.\marginpar{§ 34 odst. 1 písm. a) ZPr} Pracovní pokyny udílí
zaměstnavatel jednostranně

Může tak učinit i formou vnitřního předpisu,\marginpar{§ 305 odst. 1 ZPr}
který pro jednotlivé funkce blíže upřesňuje jednotlivé povinnosti
zaměstnance stanovené zákoníkem práce nebo vyplývající ze smlouvy.
Takovou povinností je i povinnost zaměstnance střežit\marginpar{§ 301 písm. d) ZPr}
a ochraňovat majetek zaměstnavatele před poškozením, ztrátou, zničením
a zneužitím a nejednat v rozporu s oprávněnými zájmy zaměstnavatele.
Majetkem zaměstnavatele je i zaměstnanecké dílo. Zaměstnavatel může
stanovit pravidla pro evidenci, označování a zpřístupněním zaměstnaneckých
děl v souladu s posláním školy. Jde pouze o konkretizaci zákonné povinnosti
zaměstnance, nikoliv o zavedení nové povinnosti, což je zákoníkem
práce \marginpar{§ 305 odst. 1 ZPr}zakázáno.\cite[s. 612]{Hurka2014}
Je povinností vedoucích zaměstnanců,\marginpar{§ 302 písm. f) ZPr}
aby zabezpečovali dodržování vnitřních předpisů.

V případě vysokých škol upravuje tuto problematiku typicky vnitřní
mzdový předpis.\marginpar{§ 17 odst. 1 písm. c) ZVŠ} Vnitřním
předpis ve smyslu zákona o vysokých školách je zároveň vnitřním předpisem
vydaným zaměstnavatelem podle zákoníku práce, byť obecně to pochopitelně
neplatí. Vnitřní mzdový předpis stanoví obecným způsobem okruh úkolů,
které pro zaměstnavatele plní zaměstnanci na určitých pracovních pozicích.

Příkladem může být úprava na Univerzitě Karlově v Praze.%
\footnote{VI. úplné znění vnitřního mzdového předpisu Univerzity Karlovy v Praze
ze dne 18. června 2013. Dostupné na \url{http://www.cuni.cz/UK-4343-version1-uzvimzdyuk.pdf}.%
} Tvůrčí činnost a publikace výsledků je podle vnitřního mzdového předpisu
předpokladem všech pracovních pozic uvedených v Katalogu prací pro
akademické a vědecké pracovníky. Díla vytvořená v rámci plnění této
publikační povinnosti jsou tedy zaměstnaneckými díly, neboť vznikají
ke splnění povinnosti z pracovněprávního poměru. Majetková autorská
práva k nim vykonává ze zákona zaměstnavatel (v tomto případě UK),
což je také uvedeno v příslušném opatření rektora, podle něhož přísluší
nakládání s těmito právy děkanovi.%
\footnote{Opatření rektora UK č. 7/2014 ze dne 30. 1. 2014, o nakládání s výsledky
výzkumu, vývoje a inovací na Univerzitě Karlově v Praze. \url{http://www.cuni.cz/UK-5588.html} %
}


\subsubsection{Označování}

Problém zaměstnaneckých děl je ovšem ten, že nejsou dosud konzistentně
evidována, označována a zpřístupněna, byť některé součásti UK požadují
jejich nahlášení a řadu údajů podstatných pro kvalifikaci díla jako
zaměstnaneckého lze zjistit například z Rejstříku informací o výzkumu
v informačním systému výzkumu, experimentálního vývoje a inovací (RIV).
Údaje o studijních materiálech takto dostupné nejsou. Autoři často
o statutu zaměstnaneckého díla netuší.


\subsubsection{Evidence }


\subsubsection{Zpřístupnění (licence)}

Upozorňuji také na další na problémy související s využitím zaměstnaneckých
děl, které by při přísném právním výkladu bránily bez výslovného souhlasu
UK tomu, aby např. studenti dávali své poznámky z přednášky ke sdílení
na Internet (přednášky jsou zaměstnaneckým dílem, zápisky studenta
jsou zpracováním ta­kového díla). Kdyby totiž UK přenechala práva
na přednášku jako zaměstnanecké dílo, musela by je pak zvlášť získávat
sama např. při televizním vysílání, ale také by studenti nemohli sdílet
svoje zápisky, neboť ty jsou zpracováním původní přednášky a bylo
by k tomu třeba souhlasu autora.

Problémem je v praxi právní nejistota při udělování veřejné licence
typu Creative Commons, kterou nemůže autor bez souhlasu UK podle autorského
zákona platně udělit. Navrhuji řešení otázky zaměstnaneckých děl vzniklých
na UK v souladu s výše uvedenými principy Otevřeného přístupu. 

(viz vymezení dále) jako jsou Wikiskripta fungující pod odlišnou licencí,
Univerzita by vnitřním před­pisem udělovala v souladu s licenční politikou
daného projektu licenci CC­-BY. 
\begin{enumerate}
\item 1. Zvolená licence CC-BY-SA umožňuje volné šíření za podmínky uvedení
au­tora a další zpracování díla za podmínky, že výsledné dílo bude
možné dále užít a zpracovat stejně jako původní dílo. Jde tedy o řešení
nejlépe naplňují­cí poslání UK, kterým je šíření vzdělanosti a ochrana
poznaného. 2. Univerzita i autor budou oprávněni své zaměstnanecké
dílo umístit na In­ternet, což je nutné u univerzitních projektů jako
jsou WikiSkripta případně u zpřístupnění děl na webové stránce nakladatelství
Karolinum nebo v in­ternetové knihovně. Také je budou moci vydat tiskem,
což by bylo dů­sledkem původního návrhu prof. Štípka. 3. Studenti
budou moci výsledky vlastní práce vzniklé zpracováním za­městnaneckého
díla dále sdílet s ostatními. Výsledky zaměstnaneckých děl UK (texty
a obrázky) bude možné zahrnout do největší světové encyklope­die Wikipedia,
která rovněž používá licenci CC-BY-SA. 4. Navrhované řešení také řeší
problém díla, kde práva vykonává částečně zaměstnavatel a částečně
autor. V takových případech činností autora vzniká dílo, které zpracovává
zaměstnanecké dílo pod licencí CC-BY-SA, a tak musí v souladu s bodem
1 být možné výsledné dílo šířit a zpracovat za stejných podmínek,
včetně části, ke které vykonává práva sám autor. )(V opačném případě
by se autor dopustil neoprávněného užití za­městnaneckého díla a byl
by povinen platit náhradu škody a bezdůvodné obohacení.)To vzdělávání
zcela zbytečně komplikuje a vím dokonce o situacích, kdy autor souhlas
s užitím přednášky odmí­tá. 
\end{enumerate}

\subsubsection{Způsob úpravy (vnitřní předpis, směrnice)}

\appendix

\section{Vzorové vnitřní předpisy }

V této části navrhuji několik vzorových předpisů, které je možné použít
a v původní nebo upraven podobě přijmout za vlastní. 


\subsection{Vnitřní předpis o zaměstnaneckých dílech}


\subsubsection*{Úvodní ustanovení}


\paragraph{§ 1 Předmět úpravy}

Tento vnitřní předpis se vztahuje na každé existující i budoucí zaměstnanecké
dílo ve vlastnictví školy ........ (dále jen „škola“), které je určeno
k použití pro účely výuky akademickou nebo vědeckou obcí nebo k jinému
zveřejnění v souvislosti s hlavními cíli školy. 


\paragraph{§ 2 Zaměstnanecké dílo}

Za podmínek stanovených zákonem%
\footnote{§ 58 odst. 1 zákona č. 121/2000 Sb., autorský zákon, ve znění pozdějších
předpisů.%
} je zaměstnaneckým dílem zpravidla pracovní list, učební text, cvičebnice
s úlohami, monografie, vědecký článek, posudek ke kvalifikační práci,
fotografická dokumentace školních akcí, přednáška, prezentace, dopis,
tvůrčí příspěvek do elektronického systému.


\paragraph{§ 3 Práva a povinnosti zaměstnance}

(1) Zaměstnanec má pro účely tvorby zaměstnaneckých děl právo využít
vyhrazenou část pracovní doby, využívat prostředky školy, ucházet
se o finanční podporu z rozpočtu školy, jakož i studovat a užívat
ostatní zaměstnanecká díla.

(2) Zaměstnanec je povinen ve svém zaměstnaneckém díle řádně citovat
ostatní autory, vyloučit z díla části, které je zakázáno podle zákona
zveřejnit, zajistit jeho řádné označení (§ X), zápis (§ Y), uchování
(§ Z) a zpřístupnění (§ Z). 


\subsubsection*{Označení a evidence}


\paragraph{§ 4 Označení zaměstnaneckých děl}

Označením se rozumí uvedení následujících údajů přímo v díle nebo
v doprovodné informaci:
\begin{enumerate}
\item název školy nebo jiný údaj, prostřednictvím něhož bude zřejmé, že
jde o zaměstnanecké dílo,
\item uvedení podmínek zpřístupnění díla v souladu s tímto předpisem.
\end{enumerate}
Je-li autorské dílo takto označeno a není-li prokázán opak, považuje
se za zaměstnanecké.


\paragraph{§ 5 Zápis zaměstnaneckého díla}

(1) Nejde-li o dílo zanedbatelné majetkové hodnoty, zapisují se údaje
o zaměstnaneckém díle do seznamu zaměstnaneckých děl. Seznam zaměstnaneckých
děl je informační systém, který k tomuto účelu škola provozuje. Rozsah
evidovaných údajů stanoví správce tohoto informačního systému. 

(2) Zaměstnanecké dílo se zapisuje bez zbytečného odkladu po jejich
dokončení. Nedokončené zaměstnanecké dílo se zapisuje bezprostředně
před tím, než skončí zaměstnancův pracovní poměr.


\paragraph{Zaručené uchování}

Dokončené zaměstnanecké dílo musí být uloženo v informačním systému,
kde bude zaručeno trvalé uchování jeho obsahu ve všech formátech,
ve kterých bylo vytvořeno, a odkud bude možné dílo pro účely školy
opakovaně získávat. Pro tento účel může zaměstnanec zvolit např. institucionální
repozitář školy nebo jiné důvěryhodné výzkumné anebo vědecké instituce,
projekt školy nebo jejích partnerů uchovávající obsah, případně vhodný
veřejně přístupný informační systém. Vedoucí zaměstnanec zveřejní
na pracovišti doporučené způsoby zaručeného uchování zaměstnaneckých
děl. 


\subsubsection*{Zpřístupnění}

Dílo se zveřejňuje a užívá se souhlasem školy uděleným podle tohoto
předpisu nebo příslušným orgánem školy. 

Právní otázky


\paragraph{§ Výhrada odkladu plného zpřístupnění}

V odůvodněných případech může zaměstnanec odložit plné zpřístupnění
na nezbytně nutnou dobu, nejdéle však 5 let.


\paragraph{§ 6 Veřejná licence}

(1) Základní veřejnou licencí se rozumí některá verze licence \emph{Creative
Commons -- Uveďte autora -- Zachovejte licenci, }případně\emph{ Všeobecné
veřejné licence GNU GPL.}

(2) Doplňkovou veřejnou licencí se rozumí některá verze licence \emph{Creative
Commons -- Uveďte autora. }

(3) Pokud připadá v úvahu více možných licencí, vztahuje se licenční
souhlas nebo nabídka na každou z nich.


\paragraph{§ 5 Souhlas s užitím zaměstnaneckého díla}

Není-li u díla výslovně uvedeno jinak, škola souhlasí s užitím díla
za podmínek
\begin{enumerate}
\item základní veřejné licence,
\item doplňkové veřejné licence, pokud jde o příspěvek do projektu, do něhož
se vysoká škola zapojila a který takovou licenci vyžaduje,
\end{enumerate}
Zastupování školy při udělování licence v ostatních případech upravuje
jiný vnitřní předpis.


\subsubsection*{Závěrečná ustanovení}


\paragraph{§ 8 Díla částečně zaměstnanecká}

Na zpracování zaměstnaneckého díla a na část díla, která je sama o
sobě zaměstnaneckým dílem, se použijí ustanovení tohoto předpisu o
zaměstnaneckém díle obdobně.


\paragraph{§ 9 Účinnost}

Tento vnitřní předpis nabývá účinnosti registrací Ministerstvem školství,
mládeže a tělovýchovy.


\end{document}
